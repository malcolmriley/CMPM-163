\documentclass[11pt]{article}
% PACKAGE IMPORTS
\usepackage{geometry}
\usepackage{graphicx}
\usepackage{amsmath}
\usepackage{amssymb}
\usepackage{titling}
\usepackage{listings}
\usepackage{enumitem}

\usepackage{tgschola}

% DOCUMENT COMMANDS
\newcommand{\thickline}{\noindent\rule{\textwidth}{2pt}}
\newcommand{\preparetitle}{\title{CMPM 163 Notes}\author{Malcolm Riley}\date{Winter 2019\quad\textemdash\quad\today}}
\newenvironment{topic}[1]{\section*{#1}\begin{itemize}}{\end{itemize}}
\newcommand{\inputcode}[2]{\lstinputlisting[title=\textbf{#1},label=#2]{#2}}
\newcommand{\code}[1]{\texttt{#1}}
\newcommand{\term}[1]{\textbf{#1}}
\newcommand{\quotes}[1]{``#1''}

% DOCUMENT CONFIGURATION
\geometry{letterpaper, margin=1in}
\pretitle{\thickline\begin{center}\LARGE\bfseries}
\posttitle{\end{center}}
\predate{\begin{center}}
\postdate{\end{center}\thickline\\[1em]}
\setlist[itemize,1]{label=\textbf{--}}
\lstset{basicstyle=\small\ttfamily,float=h,tabsize=4,frame=single,firstnumber=1,numbers=left,stepnumber=1,aboveskip=1em,belowskip=1em}
\preparetitle

\begin{document}
\maketitle

\begin{topic}{Audio in Unity}
	\item Audio in Unity is accomplished by the following two object components: \code{Audio Source} and \code{Audio Listener}.
	\item Typically, the \code{Audio Listener} is attached to the main camera, and in the default scene it will be set up this way.
	\item Importing audio assets works the same way as importing other assets in Unity - simple drag-n-drop.
	\item Implementing \code{GameObject} response to audio is accomplished via scripts
	\item The general idea is to populate an array of floating-point values with data from the audio spectrum. This data is then interpreted to the desired effect.
	\item The interpretation of an audio waveform is based on the principle of the \term{Fourier Transform}.
	\item Consider that an audio wave may be represented by the \term{amplitude} of a function over the domain of time.
	\item The \term{Fourier Transform} interprets the amplitude components of the waveform function at different \quotes{frequencies}; the number of discrete frequencies being sampled is the \quotes{size} of the transform. These samples are typically over a very short duration, usually on the order of milliseconds
	\item The \term{Frequency Resolution} is given by the formula $\frac{R}{S}$ where $R$ is the sample rate in \textsc{Hz}, and $S$ is the \quotes{size} of the \term{Fast Fourier Transform}
	\item The maximum measurable frequency, called the \term{Nyquist Limit}, is equal to one half of the \term{Sample Rate}.
	\item Fortunately, Unity has built-in methods for acquiring \term{Fast Fourier Transform} data from an \code{Audio Source}. On an \code{AudioSource} component, call the \code{GetSpectrumData()} method! It will populate a \code{float} array with the desired data.
	\item When performing audio analysis, there is always a tradeoff. The larger the \term{Fast Fourier Transform} size, the higher the frequency resolution, but computational expense increases. Furthermore, resolution scales over the two domains independently: the frequency domain, and the time domain.
\end{topic}

\end{document}