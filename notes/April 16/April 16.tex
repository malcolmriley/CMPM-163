\documentclass[11pt]{article}
% PACKAGE IMPORTS
\usepackage{geometry}
\usepackage{graphicx}
\usepackage{amsmath}
\usepackage{amssymb}
\usepackage{titling}
\usepackage{listings}
\usepackage{enumitem}

\usepackage{tgschola}

% DOCUMENT COMMANDS
\newcommand{\thickline}{\noindent\rule{\textwidth}{2pt}}
\newcommand{\preparetitle}{\title{CMPM 163 Notes}\author{Malcolm Riley}\date{Winter 2019\quad\textemdash\quad\today}}
\newenvironment{topic}[1]{\section*{#1}\begin{itemize}}{\end{itemize}}
\newcommand{\inputcode}[2]{\lstinputlisting[title=\textbf{#1},label=#2]{#2}}
\newcommand{\code}[1]{\texttt{#1}}
\newcommand{\term}[1]{\textbf{#1}}
\newcommand{\quotes}[1]{``#1''}

% DOCUMENT CONFIGURATION
\geometry{letterpaper, margin=1in}
\pretitle{\thickline\begin{center}\LARGE\bfseries}
\posttitle{\end{center}}
\predate{\begin{center}}
\postdate{\end{center}\thickline\\[1em]}
\setlist[itemize,1]{label=\textbf{--}}
\lstset{basicstyle=\small\ttfamily,float=h,tabsize=4,frame=single,firstnumber=1,numbers=left,stepnumber=1,aboveskip=1em,belowskip=1em}
\preparetitle

\begin{document}
\maketitle

\begin{topic}{Shader Concepts}
	\item A \term{Texel} is conceptually equivalent to a pixel, but refers to the use case of sampling from a texture.
	\item When sampling from a texture, it's possible to query the neighboring pixels
	\item In Unity, if there exists a texture called \code{\_MainTex}, there will exist a special \code{uniform float4} called \code{\_MainTex\_TexelSize} with the following members: \code{x} and \code{y} are the width and height of the texture in pixels, respectively
\end{topic}

\begin{topic}{Rendering to Other Targets}
	\item A powerful technique in shader programming is to render to an intermediate state (other than the screen), typically referred to as a \term{Offscreen Buffer}, \term{FBO}, or \term{Frame Buffer Object}. In Unity, this construct is called a \term{RenderTexture}.
	\item Instead of rendering directly to the screen, the shader renders to a special memory construct - the aforementioned buffer.
	\item This buffer object exists in the GPU's memory.
	\item This technique can be used for any number of fancy effects, such as a ``security-camera'' effect, a ``game-of-life'' effect, or an exotic ``reaction-diffusion'' effect.
\end{topic}

\begin{topic}{Ping-Pong Render Technique}
	\item Also referred to as \term{Double-Buffered Rendering}
	\item In this technique, the output of one pass in a frame will be used as the input of another pass in the same frame, or the output of one frame will be used as the input of another frame
	\item Since texture data on the GPU is read-only, it will be necessary to swap inputs between two buffer objects; hence the name of ``Ping-Pong''
	\item This method can be used to perform iterative computation on the GPU.
	\item Every sophisticated iterative technique in which it is necessary to know the output of the previous timestep will use this or a similar technique (smoke, water, clouds, fire, etc.)
	\item It is not uncommon to have multiple shaders performing different operations between each ``Ping-Pong'': Either chained per-frame, or alternating their execution
\end{topic}

\begin{topic}{Game of Life}
	\item The universe is an infinite two-dimensional orthogonal grid of square cells, each of which has two possible states: alive, and dead
	\item Each cell interacts with its eight neighbors, which are the cells that are horizontally, vertically and diagonally adjacent
	\item The behavior of these cells is governed by simple rules
\end{topic}

\begin{topic}{Unity Texture Manipulation}
	\item Unity provides a number of texture-manipulation methods\\
	\begin{tabular}{lccp{2in}}
		\textbf{Method} & \textbf{Type} & \textbf{Relative Speed} & \textbf{Explanation} \\ \hline
		\code{SetPixels}, \code{GetPixels} & CPU & Fast & Inserts or extracts data from the texture. The texture must be available on the CPU. \\
		\code{CopyTexture} & GPU & Fast & Moves data from one texture to another \\
		\code{Apply} & CPU to GPU & Slow & Copy data from the CPU to the GPU \\
		\code{ReadPixels} & GPU to CPU & Slow & Copeis texture data from GPU to the CPU \\
		\code{Blit} & GPU & Fast & Triggers rendering into an offscreen buffer (the \code{RenderTexture}) \\
	\end{tabular}
\end{topic}

\begin{topic}{Texture Formats}
	\item There are quite a few
	\item This class will concern itself primarily with \code{RGBA32}. Each \term{channel} (Red, Green, Blue, Alpha) is an 8 bit floating-point value
	\item See docs.unity3d.com/ScriptReference/TextureFormat.html for an itemized list of what Unity supports
	\item For certain computations, one may need more precision, and for that one can use \code{RGBAFloat} which offers full floating-point precision per channel
\end{topic}

\end{document}