\documentclass[11pt]{article}
% PACKAGE IMPORTS
\usepackage{geometry}
\usepackage{graphicx}
\usepackage{amsmath}
\usepackage{amssymb}
\usepackage{titling}
\usepackage{listings}
\usepackage{enumitem}

\usepackage{tgschola}

% DOCUMENT COMMANDS
\newcommand{\thickline}{\noindent\rule{\textwidth}{2pt}}
\newcommand{\preparetitle}{\title{CMPM 163 Notes}\author{Malcolm Riley}\date{Winter 2019\quad\textemdash\quad\today}}
\newenvironment{topic}[1]{\section*{#1}\begin{itemize}}{\end{itemize}}
\newcommand{\inputcode}[2]{\lstinputlisting[title=\textbf{#1},label=#2]{#2}}
\newcommand{\code}[1]{\texttt{#1}}
\newcommand{\term}[1]{\textbf{#1}}
\newcommand{\quotes}[1]{``#1''}

% DOCUMENT CONFIGURATION
\geometry{letterpaper, margin=1in}
\pretitle{\thickline\begin{center}\LARGE\bfseries}
\posttitle{\end{center}}
\predate{\begin{center}}
\postdate{\end{center}\thickline\\[1em]}
\setlist[itemize,1]{label=\textbf{--}}
\lstset{basicstyle=\small\ttfamily,float=h,tabsize=4,frame=single,firstnumber=1,numbers=left,stepnumber=1,aboveskip=1em,belowskip=1em}
\preparetitle

\begin{document}
\maketitle

\begin{topic}{Intro to Textures and Lighting}
	\item On the GPU, vertex positions are converted to texture coordinates in \term{UV Space}.
	\item \term{UV Space} is a coordinate system of floating-point values ranging between $0$ and $1$.
	\item The basic lighting model is \code{surfaceColor = emissive + ambient + diffuse + specular}. In short, just add all the values together.
	\begin{itemize}
		\item \term{Ambient Light} is the natural light inside a scene. This can be thought of as the ``sun'' - a strong, distant light source.
		\item \term{Diffuse Light} is the light that is reflected by the object in all directions. This typically only relies on the surface normal and light vector. Typically this is calculated by the expression $D = K_d \times C \times \code{max}(N \cdot L, 0)$, where $N$ is the surface normal, $L$ is the light vector, $C$ is the incoming light color, and $K_d$ is the diffuse color.
		\item \term{Emissive Light} is the light that the object itself.
		\item \term{Specular Light} is the light reflected by the ``shiny bits'' of the object. This is typically calculated by the expression $S = K_s \times C \times f \times \code{max}(N \cdot H, 0)^{s}$ where $K_s$ is the specular color, $C$ is the incoming light color, $f$ is $1$ or $0$ depending on whether the polygon is facing the camera, $N$ is the surface normal, and $H$ is the normalized vector halfway between the normalized viewpoint vector and $L$.
	\end{itemize}
	\item \term{Flat Shading} is a shading method such that the shading is not interpolated in the raster process, but is constant per polygon normal.
	\item \term{Gouraud Shading} is a method of shading in which the lighting calculation is performed in the vertex shader rather than in the fragment shader.
\end{topic}

\begin{topic}{A Closer Look at Lighting}
	\item Table of Component Expressions: \\
	\begin{tabular}{llp{3in}}
		\textbf{Component} & \textbf{Expression} & \textbf{Legend} \\ \hline
		Ambient & $K_a \times C$ & $K_a$ = Material Ambient Reflectance; $C$ = Incoming Light Color \\
		Diffuse & $K_d \times C \times \code{max}(N \cdot L, 0)$ & $K_d$ = Material diffuse color; $C$ = Incoming diffuse light color; $N$ = Normalized surface normal; $L$ = Normalized vector facing light source \\
		Specular & $$ & \\
	\end{tabular}
\end{topic}

\end{document}