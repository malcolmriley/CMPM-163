\documentclass[11pt]{article}
% PACKAGE IMPORTS
\usepackage{geometry}
\usepackage{graphicx}
\usepackage{amsmath}
\usepackage{amssymb}
\usepackage{titling}
\usepackage{listings}
\usepackage{enumitem}

\usepackage{tgschola}

% DOCUMENT COMMANDS
\newcommand{\thickline}{\noindent\rule{\textwidth}{2pt}}
\newcommand{\preparetitle}{\title{CMPM 163 Notes}\author{Malcolm Riley}\date{Winter 2019\quad\textemdash\quad\today}}
\newenvironment{topic}[1]{\section*{#1}\begin{itemize}}{\end{itemize}}
\newcommand{\inputcode}[2]{\lstinputlisting[title=\textbf{#1},label=#2]{#2}}
\newcommand{\code}[1]{\texttt{#1}}
\newcommand{\term}[1]{\textbf{#1}}
\newcommand{\quotes}[1]{``#1''}

% DOCUMENT CONFIGURATION
\geometry{letterpaper, margin=1in}
\pretitle{\thickline\begin{center}\LARGE\bfseries}
\posttitle{\end{center}}
\predate{\begin{center}}
\postdate{\end{center}\thickline\\[1em]}
\setlist[itemize,1]{label=\textbf{--}}
\lstset{basicstyle=\small\ttfamily,float=h,tabsize=4,frame=single,firstnumber=1,numbers=left,stepnumber=1,aboveskip=1em,belowskip=1em}
\preparetitle

\begin{document}
\maketitle

\begin{topic}{Miscellaneous}
	\item Quiz next week: Tuesday, April 30th - mix of theoretical and code questions (Terminology and GLSL commands)
	\item Careful when using Unity's default lighting system, it may be doing multiple lighting passes without telling you - to resolve this issue, activate blending in your shader; alternately, pass point light positions directly and use these positions to calculate lighting
\end{topic}

\begin{topic}{Using Images for Vertex Displacement}
	\item A common use-case is the generation of \term{Height Maps}. By detecting vertex height in a shader, it's trivially simple to map different textures to a mesh depending on y-displacement.
	\item One technique is to pass a height texture to the shader.
	\item Unity provides facility to directly edit meshes in script.
	\item Unity offers the \code{RecalculateNormals()} method on \code{Mesh} objects to automatically set normals on that mesh.
	\item In certain circumstances, the \code{tex2D()} lookup is not available in the vertex shader. However, \code{tex2Dlod()} may be available. This is because \code{tex2D()} attempts to automatically set the mip level, which isn't necessarily available before rasterization.
\end{topic}

\begin{topic}{Noise and Procedural Generation in Shaders}
	\item There are different types of noise, which are different in terms of aggregate structure (such as periodicity)
	\item In certain types of noise, individual pixels may have some relationship to their neighboring pixels.
	\item \term{Perlin Noise} was originally developed by Ken Perlin as he was working on the original Tron movie. His task was to create a new kind of naturalistic visual effect computationally.
	\item The general idea behind \term{Perlin Noise} is to create \quotes{bins} for which numbers are generated randomly; these \quotes{bins} are then subdivided, and the values are interpolated between.
	\item In noise generation, the term \term{Octave} refers to the practice of combining multiple noise waveforms to create a more sophisticated waveform. 
\end{topic}

\end{document}