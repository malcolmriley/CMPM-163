\documentclass[11pt]{article}
% PACKAGE IMPORTS
\usepackage{geometry}
\usepackage{graphicx}
\usepackage{amsmath}
\usepackage{amssymb}
\usepackage{titling}
\usepackage{listings}
\usepackage{enumitem}

\usepackage{tgschola}

% DOCUMENT COMMANDS
\newcommand{\thickline}{\noindent\rule{\textwidth}{2pt}}
\newcommand{\preparetitle}{\title{CMPM 163 Notes}\author{Malcolm Riley}\date{Winter 2019\quad\textemdash\quad\today}}
\newenvironment{topic}[1]{\section*{#1}\begin{itemize}}{\end{itemize}}
\newcommand{\inputcode}[2]{\lstinputlisting[title=\textbf{#1},label=#2]{#2}}
\newcommand{\code}[1]{\texttt{#1}}
\newcommand{\term}[1]{\textbf{#1}}
\newcommand{\quotes}[1]{``#1''}

% DOCUMENT CONFIGURATION
\geometry{letterpaper, margin=1in}
\pretitle{\thickline\begin{center}\LARGE\bfseries}
\posttitle{\end{center}}
\predate{\begin{center}}
\postdate{\end{center}\thickline\\[1em]}
\setlist[itemize,1]{label=\textbf{--}}
\lstset{basicstyle=\small\ttfamily,float=h,tabsize=4,frame=single,firstnumber=1,numbers=left,stepnumber=1,aboveskip=1em,belowskip=1em}
\preparetitle

\begin{document}
\maketitle

\begin{topic}{Detecting Input in Unity}
	\item All script-based input detection in Unity is handled via the \code{Input} class in the \code{UnityEngine} namespace.
	\item To query mouse position, use \code{Input.mouseposition.x} and \code{Input.mouseposition.y}
	\item To query for a particular keypress, use \code{Input.GetKey("name")} where \code{name} is the name of the key to query for.
\end{topic}

\begin{topic}{Textures and Shaders}
	\item Texture coordinates, the \term{UV} of the texture, are a pair of floating point values $u$ and $v$ in the range $[0, 1]$
	\item Unity automatically generates UV coordinates for basic game objects, as will most 3D modeling programs.
	\item The mapping between the UV coordinates of a particular polygon in a mesh and the corresponding UV coordinates of the input texture file determines how that particular polygon will be textured in the shader.
	\item Conceptually, all fragments in the fragment shader are processed simultaneously (and thus, independently of the others)
\end{topic}

\begin{topic}{Edge Detection in Shaders}
	\item See: Sobel Operator and Edge Detection on Wikipedia
	\item The fundamental idea is to examine the image for discontinuity between color of the current pixel and its neighbors. If there is a high discontinuity, color as the edge color.
	\item There are different types of edge detection, including comparison of normals, etc.
\end{topic}

\end{document}