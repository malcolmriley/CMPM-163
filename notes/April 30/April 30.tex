\documentclass[11pt]{article}
% PACKAGE IMPORTS
\usepackage{geometry}
\usepackage{graphicx}
\usepackage{amsmath}
\usepackage{amssymb}
\usepackage{titling}
\usepackage{listings}
\usepackage{enumitem}

\usepackage{tgschola}

% DOCUMENT COMMANDS
\newcommand{\thickline}{\noindent\rule{\textwidth}{2pt}}
\newcommand{\preparetitle}{\title{CMPM 163 Notes}\author{Malcolm Riley}\date{Winter 2019\quad\textemdash\quad\today}}
\newenvironment{topic}[1]{\section*{#1}\begin{itemize}}{\end{itemize}}
\newcommand{\inputcode}[2]{\lstinputlisting[title=\textbf{#1},label=#2]{#2}}
\newcommand{\code}[1]{\texttt{#1}}
\newcommand{\term}[1]{\textbf{#1}}
\newcommand{\quotes}[1]{``#1''}

% DOCUMENT CONFIGURATION
\geometry{letterpaper, margin=1in}
\pretitle{\thickline\begin{center}\LARGE\bfseries}
\posttitle{\end{center}}
\predate{\begin{center}}
\postdate{\end{center}\thickline\\[1em]}
\setlist[itemize,1]{label=\textbf{--}}
\lstset{basicstyle=\small\ttfamily,float=h,tabsize=4,frame=single,firstnumber=1,numbers=left,stepnumber=1,aboveskip=1em,belowskip=1em}
\preparetitle

\begin{document}
\maketitle

\begin{topic}{Environment Mapping}
	\item Also known as \term{Cubemapping}
	\item Basically, this is the practice of shading the interior of a cube with six seamless textures
	\item Used for creating skyboxes. Shaders can do neat things with skyboxes, such as reflecting them.
	\item Skyboxes are typically implemented as being \quotes{infinitely far away}, such that the character's motion has no impact on the relative positioning of the skybox
	\item The typical UV mapping for a \term{cubemap} is spherical, this allows a cube to be used in place of a sphere
	\item Texture coordinates of the \term{cubemap} are three dimensional, not two dimensional; however these coordinates map onto the flattened cubic plane
	\item In Unity, the \code{Skybox Material} field may be set via the \code{Lighting} settings. This is how special shaders are applied to the skybox.
	\item Cg uses \code{samplerCUBE()} to sample from \term{cubemap} (In contrast with \code{sampler2D()}
	\item Cg/HLSL have built-in \code{reflect()} and \code{refract()} functions to return the ray of reflection/refraction of a vector based on the mesh surface normal. \code{refract()} requires an additional \code{float} interpreted as an angle of refraction.
	\item In real-world refractive media, different spectra have different ratios of refraction. This effect is called \term{Chromatic Dispersion} and is the essential characteristic of the function of prisms. This can be accomplished in shaders by doing different \code{texCUBE} lookups per chroma. This effect is typically quite subtle in real-world materials, so use sparingly
\end{topic}

\end{document}