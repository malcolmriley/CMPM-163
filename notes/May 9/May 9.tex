\documentclass[11pt]{article}
% PACKAGE IMPORTS
\usepackage{geometry}
\usepackage{graphicx}
\usepackage{amsmath}
\usepackage{amssymb}
\usepackage{titling}
\usepackage{listings}
\usepackage{enumitem}

\usepackage{tgschola}

% DOCUMENT COMMANDS
\newcommand{\thickline}{\noindent\rule{\textwidth}{2pt}}
\newcommand{\preparetitle}{\title{CMPM 163 Notes}\author{Malcolm Riley}\date{Winter 2019\quad\textemdash\quad\today}}
\newenvironment{topic}[1]{\section*{#1}\begin{itemize}}{\end{itemize}}
\newcommand{\inputcode}[2]{\lstinputlisting[title=\textbf{#1},label=#2]{#2}}
\newcommand{\code}[1]{\texttt{#1}}
\newcommand{\term}[1]{\textbf{#1}}
\newcommand{\quotes}[1]{``#1''}

% DOCUMENT CONFIGURATION
\geometry{letterpaper, margin=1in}
\pretitle{\thickline\begin{center}\LARGE\bfseries}
\posttitle{\end{center}}
\predate{\begin{center}}
\postdate{\end{center}\thickline\\[1em]}
\setlist[itemize,1]{label=\textbf{--}}
\lstset{basicstyle=\small\ttfamily,float=h,tabsize=4,frame=single,firstnumber=1,numbers=left,stepnumber=1,aboveskip=1em,belowskip=1em}
\preparetitle

\begin{document}
\maketitle

\begin{topic}{Guest Speaker: Javier Villegas}
	\item There is an approach to audio processing called \term{analysis and synthesis} - Using the parameterized results of the analysis of of a signal to synthesize a result from another.
	\item This approach can also be applied to video processing
	\item Fundamentally, this approach takes advantage of the human brain's penchant for detecting patterns
	\item There also exists the problem of \term{temporal coherence} - There needs to be some interpolation or smoothing between two different frames in a real-time video, due to signal noise or other video artifacts
	\item Solving the \term{temporal coherence} problem is often a question of weighing sub-optimal matching - how much informational decoherence from the original source is acceptable
	\item Another solution is a \term{gradient-based approach}, an image is fed into a surface gradient evaluator, which is then interpreted as a vector field. This vector field can be smoothed and interpolated to achieve the desired results
	\item Furthermore, a \term{template-matching} algorithm can be used to find a best-match from a map of known vectors, and render the value mapped to that key bucket
	\item Global or local histogram equalization of the luminance of the image can be used to identify regions of interest
	\item Derivatives of the movement delta of individual points-of-interest can also be used for interesting effects
	\item The center of mass of a field of pixels can be tracked and used as a data point for influencing other effects
\end{topic}

\end{document}