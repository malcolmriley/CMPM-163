\documentclass[11pt]{article}
% PACKAGE IMPORTS
\usepackage{geometry}
\usepackage{graphicx}
\usepackage{amsmath}
\usepackage{amssymb}
\usepackage{titling}
\usepackage{listings}
\usepackage{enumitem}

\usepackage{tgschola}

% DOCUMENT COMMANDS
\newcommand{\thickline}{\noindent\rule{\textwidth}{2pt}}
\newcommand{\preparetitle}{\title{CMPM 163 Notes}\author{Malcolm Riley}\date{Winter 2019\quad\textemdash\quad\today}}
\newenvironment{topic}[1]{\section*{#1}\begin{itemize}}{\end{itemize}}
\newcommand{\inputcode}[2]{\lstinputlisting[title=\textbf{#1},label=#2]{#2}}
\newcommand{\code}[1]{\texttt{#1}}
\newcommand{\term}[1]{\textbf{#1}}
\newcommand{\quotes}[1]{``#1''}

% DOCUMENT CONFIGURATION
\geometry{letterpaper, margin=1in}
\pretitle{\thickline\begin{center}\LARGE\bfseries}
\posttitle{\end{center}}
\predate{\begin{center}}
\postdate{\end{center}\thickline\\[1em]}
\setlist[itemize,1]{label=\textbf{--}}
\lstset{basicstyle=\small\ttfamily,float=h,tabsize=4,frame=single,firstnumber=1,numbers=left,stepnumber=1,aboveskip=1em,belowskip=1em}

\usepackage{ textcomp } % For \textrightarrow

\preparetitle

\begin{document}
\maketitle

\begin{topic}{Insights from the Animation Industry}
	\item Animation and film studios are increasingly using real-time technology to produce their works
	\item The traditional animation pipeline had avoided real-time technology as it was not \quotes{accurate enough} to achieve the desired results
	\item A \term{Lighting TD} will work with the primary director to establish narrative and visual storytelling through shot emotion and tone; they will typically also integrate VFX and physical simulations with existing models (hair, cloth, etc) in addition to homogenizing the inter-scene lighting aesthetic
	\item The construction of a lighted asset requires constant weighing of the dual considerations between \quotes{realism} and \quotes{believability}. A \quotes{realistic} effect is not always \quotes{believable}, and vice-versa
	\item The \term{Dappled Lighting} technique produces a \quotes{filtered distant-light effect} -- imagine sunlight filtered through a tree canopy: the lighting effect produced by this environment is analogous.
	\item In some situations, it may be advantageous to light different objects within the same scene using different sets of lights - some lights only affect certain objects, etc.
	\item Lighting is more than adding realism or aesthetic quality : it can also be used as a design principle to construct scene geometry, establishing visual hierarchy, points of focus, etc. -- effectively visual communication
\end{topic}

\begin{topic}{Insights from the Games Industry}
	\item As a \term{Level Designer}, one will work as a director over a particular level of a game -- deciding the layout, sequence of events, gestalt aesthetic, intended player dynamics, tempo and cadence actions, overall flow etc.
	\item To the question of whether working on violent/gory games affects you: \quote{You take home with you what you work on -- you'll be working on it 8 hours a day and it sticks to you.}
	\item It's a good idea to get the fundamentals of the game down on paper before anything else -- try to avoid rebuilding the core once the \quotes{real} assets start rolling in.
	\item Ideally, the \quotes{paper-map} stage is the point at which the designer finds the \quotes{broken} parts of the game. It's typically more expensive (in terms of time and money) to fix once the system has been built.
	\item Typical workflow of level design: Concept \textrightarrow Paper Map \textrightarrow White Box Digital Model \textrightarrow Actual Implementation
	\item In a perfect world, all the mechanics of the game is \textit{completed} during the White Box phase, before any final assets are added.
\end{topic}

\begin{topic}{Humane Technology}
	\item The fundamental questions of \term{Humane Technology} is to ask, \quotes{What do humans stand to gain from digital technology?}, or \quotes{How can digital technology be used to enhance the human experience?} and \quotes{How can digital technology increase inclusiveness and accessibility?}
	\item Principles of humane technology:
	\begin{itemize}
		\item Human-centric design
		\item Ethical consideration
		\item Inclusivity, and Conscientiousness
	\end{itemize}
	\item Basically, bring digital stuff back into the physical world as much as possible because that's good for humans
\end{topic}

\end{document}