\documentclass[11pt]{article}
% PACKAGE IMPORTS
\usepackage{geometry}
\usepackage{graphicx}
\usepackage{amsmath}
\usepackage{amssymb}
\usepackage{titling}
\usepackage{listings}
\usepackage{enumitem}

\usepackage{tgschola}

% DOCUMENT COMMANDS
\newcommand{\thickline}{\noindent\rule{\textwidth}{2pt}}
\newcommand{\preparetitle}{\title{CMPM 163 Notes}\author{Malcolm Riley}\date{Winter 2019\quad\textemdash\quad\today}}
\newenvironment{topic}[1]{\section*{#1}\begin{itemize}}{\end{itemize}}
\newcommand{\inputcode}[2]{\lstinputlisting[title=\textbf{#1},label=#2]{#2}}
\newcommand{\code}[1]{\texttt{#1}}
\newcommand{\term}[1]{\textbf{#1}}
\newcommand{\quotes}[1]{``#1''}

% DOCUMENT CONFIGURATION
\geometry{letterpaper, margin=1in}
\pretitle{\thickline\begin{center}\LARGE\bfseries}
\posttitle{\end{center}}
\predate{\begin{center}}
\postdate{\end{center}\thickline\\[1em]}
\setlist[itemize,1]{label=\textbf{--}}
\lstset{basicstyle=\small\ttfamily,float=h,tabsize=4,frame=single,firstnumber=1,numbers=left,stepnumber=1,aboveskip=1em,belowskip=1em}
\preparetitle

\begin{document}
\maketitle

\begin{topic}{Shader Case Studies}
	\item It is possible to reconstruct the world coordinates of a particular fragment in Unity by querying the depth buffer.
	\item To create a gradient from within a shader, simply use the \code{lerp()} function between two \code{float4}.
	\item A common technique is to encode operational data into a texture, sampling that texture within the shader using \code{tex2D()}. For instance, querying the \code{r} and \code{g} values for use as an offset for fragments.
	\item When using a lookup texture for screen-space effects, there is the risk that screens with different aspect ratio will cause stretching. Correcting for this may involve platform-specific scripting code.
	\item Check out \quotes{Makin' stuff look good: Stealth games' XRay Vision} and \quotes{Beyond Turing - Ray Tracing and the Future of Computer Graphics}
\end{topic}

\end{document}