\documentclass[11pt]{article}
% PACKAGE IMPORTS
\usepackage{geometry}
\usepackage{graphicx}
\usepackage{amsmath}
\usepackage{amssymb}
\usepackage{titling}
\usepackage{listings}
\usepackage{enumitem}

\usepackage{tgschola}

% DOCUMENT COMMANDS
\newcommand{\thickline}{\noindent\rule{\textwidth}{2pt}}
\newcommand{\preparetitle}{\title{CMPM 163 Notes}\author{Malcolm Riley}\date{Winter 2019\quad\textemdash\quad\today}}
\newenvironment{topic}[1]{\section*{#1}\begin{itemize}}{\end{itemize}}
\newcommand{\inputcode}[2]{\lstinputlisting[title=\textbf{#1},label=#2]{#2}}
\newcommand{\code}[1]{\texttt{#1}}
\newcommand{\term}[1]{\textbf{#1}}
\newcommand{\quotes}[1]{``#1''}

% DOCUMENT CONFIGURATION
\geometry{letterpaper, margin=1in}
\pretitle{\thickline\begin{center}\LARGE\bfseries}
\posttitle{\end{center}}
\predate{\begin{center}}
\postdate{\end{center}\thickline\\[1em]}
\setlist[itemize,1]{label=\textbf{--}}
\lstset{basicstyle=\small\ttfamily,float=h,tabsize=4,frame=single,firstnumber=1,numbers=left,stepnumber=1,aboveskip=1em,belowskip=1em}
\preparetitle

\begin{document}
\maketitle

\begin{topic}{Screen-Space Effects}
	\item \term{Screen-Space Effects} are textures captured or intercepted from the main render camera before being altered and re-rendered to the original target. \term{Post-Processing} is a type of \term{Screen-Space Effect}.
	\item In Unity, \code{MonoBehavior} features the \code{OnRenderImage} method, which can be overridden to perform these types of effects.
	\item When performing \term{Screen-Space Effects}, it is typically not permitted to alter the vertices within the vertex shader. Unity provides a default passthrough vertex shader which can be invoked using the compiler statement \code{\#pragma vertex vert\_img}.
	\item The human eye is more sensitive to changes in certain spectra, so it is occasionally beneficial to weight the luminosity of color values accordingly
	\item The \code{[ExecuteInEditMode]} class attribute allows \term{Screen-Space Effects} to be executed even if Play Mode is not active.
	\item In most shader languages, a \code{float} is a full 32-bit floating point value, whereas \code{fixed} is a 8 bit fixed-point value.
	\item Sometimes making a special effect isn't more complex, it just has more ``stuff''
	\item \term{Screen-Space Effects} are an easy way to give a distinctive look to an entire scene
\end{topic}

\end{document}