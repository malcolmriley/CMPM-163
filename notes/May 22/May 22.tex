\documentclass[11pt]{article}
% PACKAGE IMPORTS
\usepackage{geometry}
\usepackage{graphicx}
\usepackage{amsmath}
\usepackage{amssymb}
\usepackage{titling}
\usepackage{listings}
\usepackage{enumitem}

\usepackage{tgschola}

% DOCUMENT COMMANDS
\newcommand{\thickline}{\noindent\rule{\textwidth}{2pt}}
\newcommand{\preparetitle}{\title{CMPM 163 Notes}\author{Malcolm Riley}\date{Winter 2019\quad\textemdash\quad\today}}
\newenvironment{topic}[1]{\section*{#1}\begin{itemize}}{\end{itemize}}
\newcommand{\inputcode}[2]{\lstinputlisting[title=\textbf{#1},label=#2]{#2}}
\newcommand{\code}[1]{\texttt{#1}}
\newcommand{\term}[1]{\textbf{#1}}
\newcommand{\quotes}[1]{``#1''}

% DOCUMENT CONFIGURATION
\geometry{letterpaper, margin=1in}
\pretitle{\thickline\begin{center}\LARGE\bfseries}
\posttitle{\end{center}}
\predate{\begin{center}}
\postdate{\end{center}\thickline\\[1em]}
\setlist[itemize,1]{label=\textbf{--}}
\lstset{basicstyle=\small\ttfamily,float=h,tabsize=4,frame=single,firstnumber=1,numbers=left,stepnumber=1,aboveskip=1em,belowskip=1em}
\usepackage{hyperref}
\preparetitle

\begin{document}
\maketitle

\begin{topic}{Volumetric Effects and Models}
	\item A \term{Model} is a representation, abstraction, or process of a given system that may or may not employ simplification, generalization, depending on the intended accuracy of the model.
	\item \term{Models} are useful because nature is mostly chaotic, and these constructs permit us to understand, find or create order, build a hierarchy of understanding, and eventually uncover a more universal pattern.
	\item \term{Volumetric Media} is a term that refers to physical volumes that have effects on transiting light.
	\item In ordinary life, the effects of \term{Volumetric Media} are not always apparent as many physical objects appear opaque or non-translucent.
	\item Density is typically the greatest effector when it comes to calculating volumetric effects.
	\item It is possible to distinguish between \term{Physically Correct} models and \term{Empirically Correct} models. The former is evaluated based on real principles, whereas the latter is merely intended to mimic the phenomena as they are observed.
	\item The amount of light passing through a fragment can be found by taking the integral of the product reflectance with the incoming light for all directions.
	\item The fundamental problem of light transport is that light transport has a fractal nature: light is reflected, refracted, and re-emitted by surfaces, so calculating the light hitting a surface from all direction also entails calculating previous iterations of the same equation.
	\item The concept of \term{Radiative Transfer} is an abstract model representing the aggregate transfer of light of the previous problem
	\item There are two principled methods of solving the \term{Radiative Transfer} problem: \term{Stochastic/Probabilistic} or \term{Deterministic}.
	\item Within the \term{Stochastic} method is the approach of \term{Photon Mapping} - individual \quotes{photon} interactions are modeled and stored as they occur, whereupon a raymarch is employed to determine which interactions would be visible to the viewer
	\item Within the \term{Deterministic} approach, volumes are broken into cells and the individual cells perform light interactions upon each other.
	\item \term{Light Scattering} is the gradual loss of illumination coherence throughout a substance
\end{topic}

\begin{topic}{Methodology of Volumetric Effects Rendering}
	\item A typical approach to soft volumetrics is to keep a 3D texture representing discretized densities of an object.
	\item Instead of solving the global transport problem, it is possible to employ \term{Principal Ordinate Propagation} -- effectively, perform an iterated series of discretized ordinate solutions per light source, then sum the results.
	\item For a scene with many lights, they may be separated into \term{Distant Lights}, \term{Point Lights}, and \term{Virtual Point Lights}. Each will generate a light propagation lattice for the final calculation
	\item \term{Screen-Space Scattering} is an optimization that is applicable to sparse environments; it is an approximation that uses only the image-space depth to apply a volumetric media calculation to the entire screen space. Basically it assumes that there is a uniform volumetric media that encompasses the viewer and entire scene.
	\item Check out Inigo Quilez: \url{iquilezles.org}
\end{topic}

\end{document}