\documentclass[11pt]{article}
% PACKAGE IMPORTS
\usepackage{geometry}
\usepackage{graphicx}
\usepackage{amsmath}
\usepackage{amssymb}
\usepackage{titling}
\usepackage{listings}
\usepackage{enumitem}

\usepackage{tgschola}

% DOCUMENT COMMANDS
\newcommand{\thickline}{\noindent\rule{\textwidth}{2pt}}
\newcommand{\preparetitle}{\title{CMPM 163 Notes}\author{Malcolm Riley}\date{Winter 2019\quad\textemdash\quad\today}}
\newenvironment{topic}[1]{\section*{#1}\begin{itemize}}{\end{itemize}}
\newcommand{\inputcode}[2]{\lstinputlisting[title=\textbf{#1},label=#2]{#2}}
\newcommand{\code}[1]{\texttt{#1}}
\newcommand{\term}[1]{\textbf{#1}}
\newcommand{\quotes}[1]{``#1''}

% DOCUMENT CONFIGURATION
\geometry{letterpaper, margin=1in}
\pretitle{\thickline\begin{center}\LARGE\bfseries}
\posttitle{\end{center}}
\predate{\begin{center}}
\postdate{\end{center}\thickline\\[1em]}
\setlist[itemize,1]{label=\textbf{--}}
\lstset{basicstyle=\small\ttfamily,float=h,tabsize=4,frame=single,firstnumber=1,numbers=left,stepnumber=1,aboveskip=1em,belowskip=1em}
\preparetitle

\begin{document}
\maketitle

\begin{topic}{Homework Tips}
	\item Even if a build is provided it is useful to provide a download to the raw project - sometimes the builds don't work on specific platforms!
	\item Try to do a WebGL build \textit{wherever possible} - it's quicker and easier for the TAs
\end{topic}

\begin{topic}{Visual Effect Case Studies}
	\item \textbf{River VFX:}
	\begin{itemize}
		\item Potentially adjusting the UV of the texture based on angle of surface normals, in order to provide the illusion of the water accelerating over the fall
		\item Water surface has two textures with different offset-over-time speeds
	\end{itemize}
	\item \textbf{Toon Shaded Pond:}
	\begin{itemize}
		\item Standard toon shader
		\item Lighting appears to be baked into the texture
		\item Some easing function on the "bob" effect - could be implemented as vertex displacement shader but more likely implemented CPU-side
		\item Foam around the edges potentially the result of some difference between noise textures
	\end{itemize}
	\item \textbf{Unlit Waterfall:} 
	\begin{itemize}
		\item Noise around edges of water
		\item Particles at impact point of water column
		\item Some kind of toon effect on solid objects
		\item Possibly stretched Perlin noise difference function on water column
	\end{itemize}
	\item Check out: Harry Halisavakis - Technically Art
\end{topic}

\begin{topic}{Unity Shader Graph}
	\item It's a new method for creating shaders that doesn't require writing code
	\item Nodes-and-strings graph-based interface for controlling and creating new shaders
	\item Connect the inputs and outputs of a library of functional nodes to produce interesting results
	\item It may still be useful to understand shader programming internals, to understand how individual nodes will function
\end{topic}

\end{document}