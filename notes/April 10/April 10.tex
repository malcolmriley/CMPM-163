\documentclass[11pt]{article}
% PACKAGE IMPORTS
\usepackage{geometry}
\usepackage{graphicx}
\usepackage{amsmath}
\usepackage{amssymb}
\usepackage{titling}
\usepackage{listings}
\usepackage{enumitem}

\usepackage{tgschola}

% DOCUMENT COMMANDS
\newcommand{\thickline}{\noindent\rule{\textwidth}{2pt}}
\newcommand{\preparetitle}{\title{CMPM 163 Notes}\author{Malcolm Riley}\date{Winter 2019\quad\textemdash\quad\today}}
\newenvironment{topic}[1]{\section*{#1}\begin{itemize}}{\end{itemize}}
\newcommand{\inputcode}[2]{\lstinputlisting[title=\textbf{#1},label=#2]{#2}}
\newcommand{\code}[1]{\texttt{#1}}
\newcommand{\term}[1]{\textbf{#1}}
\newcommand{\quotes}[1]{``#1''}

% DOCUMENT CONFIGURATION
\geometry{letterpaper, margin=1in}
\pretitle{\thickline\begin{center}\LARGE\bfseries}
\posttitle{\end{center}}
\predate{\begin{center}}
\postdate{\end{center}\thickline\\[1em]}
\setlist[itemize,1]{label=\textbf{--}}
\lstset{basicstyle=\small\ttfamily,float=h,tabsize=4,frame=single,firstnumber=1,numbers=left,stepnumber=1,aboveskip=1em,belowskip=1em}
\preparetitle

\begin{document}
\maketitle

\begin{topic}{Shader Fundamentals}
	\item The \term{Vertex Shader} is the program that alters the vertices of a mesh in some way.
	\item Vertices -> Primitive Assembly -> Vertex Shader -> Rasterization -> 
	\item \term{Rasterization} is the process of turning connected vertices into fragments by determining which pixel regions of the render target are ``covered'' by the triangles.
	\item \term{Rasterization} is a fixed function of the GPU and currently cannot be programmed.
	\item The \code{\#pragma} directive informs the compiler about a particular shader.
\end{topic}

\begin{topic}{Shaders in Unity}
	\item \term{ShaderLab} is basically a Unity wrapper for Cg, GLSL and HLSL shader programs.
	\item In Unity, Shaders are applied to Materials, and Materials are applied to Objects; there is typically a 1:1:1 correspondence between these.
	\item \term{ShaderLab} syntax is basically a list of attributes and flags, with fields for shader code.
\end{topic}

\end{document}